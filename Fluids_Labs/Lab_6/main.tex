\title{CFD laboratory 4\\Flow around a circular cylinder in the sub-critical regime}
\author{
        Sergio M. Vanegas A.\\
        Francesco de Pas\\
                Department of Mathematics\\
        Polimi---Politecnico di Milano\\
        Milano, Italia
}
\date{\today}

\documentclass[12pt]{article}

\usepackage{amsmath}
\usepackage{graphicx}
\usepackage{siunitx}

\begin{document}
\maketitle

\begin{abstract} 
        The fourth test case is the flow around a circular cylinder in the sub-critical regime. The cylinder is considered with infinite height to avoid side effects. The flow develops from the free-stream profile $U_\infty$ in the form of a laminar boundary layer that, after separation at about 82° from the leading edge, generates a turbulent wake with some periodic behavior. The phenomenon is inherently unsteady even at the macro-scale and therefore it requires solving the U-RANS, which might be computationally demanding even for a 2D problem. As a consequence it is here performed the grid independence study for a steady state (RANS) model, and then use the same grid settings for the subsequent U-RANS simulation.\cite{FL:06}
 
        \begin{figure}[!ht]
                \includegraphics[width=\textwidth]{Flow_Sketch.png}
                \centering
                \caption{}
                \label{fig:flow_sketch}
        \end{figure}
 
 
\end{abstract}

\section{Introduction}

        In the sub-critical regime, the laminar boundary layer developing over the walls of the cylinder separates at about 82° from the front stagnation point, and a large, turbulent wake generates downstream. The pressure distribution over the walls of the cylinder, shown in Figure~\ref{fig:p_d}, agrees with the potential flow solution only in the front part of the body. The separation point is just at the beginning of the region of adverse pressure gradient, and it can be easily recognized in the figure since the wall pressure in the wake region is broadly uniform. Additionally, the wall shear stress is zero at the point of separation.


        \begin{figure}[!ht]
                \includegraphics[width=\textwidth]{Pressure_Distribution.png}
                \centering
                \caption{}
                \label{fig:p_d}
        \end{figure}

        The drag coefficient of a circular cylinder with infinite length is defined as: $$C_D = \frac{\frac{F_D}{H}}{\frac{1}{2} \rho \left(\frac{A_D}{H}\right) U_\infty^2} = \frac{\frac{F_D}{H}}{\frac{1}{2} \rho \left(\frac{D_C H}{H}\right) U_\infty^2} = \frac{\frac{F_D}{H}}{\frac{1}{2} \rho D_C U_\infty^2}$$ and it is a function of $\text{Re}_D$ and the relative roughness $\frac{s}{D_C}$. In the sub-critical regime, $C_D$ is nearly constant with $\text{Re}_D$, and it is not much affected by the roughness. Such constant value is around 1.2, as is it evident from Figure~\ref{fig:dragtrend}.

        \begin{figure}[!ht]
                \includegraphics[width=\textwidth]{DragCoefficient_Trend.png}
                \centering
                \caption{}
                \label{fig:dragtrend}
        \end{figure}

        Finally, the dimensionless Strouhal number quantifies the characteristic frequency of the turbulent wake, \textit{f},and it is defined as $St = \frac{f*D_C}{U_\infty}$. The paper by Fey et al. (1998) provides a correlation to estimate \textit{St} as a function of $\text{Re}_D$, according to which, in the sub-critical regime, \textit{St} varies between 0.185 and 0.21 Figure~\ref{fig:strouhal}.


 \begin{figure}[!ht]
                \includegraphics[width=\textwidth]{Strouhal.png}
                \centering
                \caption{}
                \label{fig:strouhal}
        \end{figure}


        The configuration of the problem is as follows:
        \begin{itemize}
                \item Diameter \( D_c = 0.06 \: m \),
                \item Free-stream velocity \( U_\infty = 0.4 \: m/s \),
                \item Bulk velocity \( U_b = 5 \: mm/s \),
                \item Fluid: Water at \( 20^{\circ}C \; \rho = 998.23 \: kg/m^3\;( \mu=1.006*10^{-6} \:m^2/s\)).
        \end{itemize}


        \paragraph{Outline}
        The remainder of the report is organized as follows: Section~\ref{sec:Steady-state precursor} provides some suitable results concerning the Grid Independence Study performed on RANS solutions; Section~\ref{sec:URAN} instead makes use of URANS in order to focus on the temporal evolution  of the process.
      
\section{Steady-state precursor} \label{sec:Steady-state precursor}

        The following Grid-Independence study is applied on RANS solution.   Despite in principle RANS solutions do not provide a trustful representation of the physical phenomenon under investigation, this choice can been consider a practical compromise to face the heavy URANS  computational cost. The variables under investigation are:  the distributions of wall pressure and wall shear stress, the drag coefficient and the position of the separation point, inferred from the wall pressure and the wall shear stresses.In particular the separation point was inferred by the wall pressure approximating the wall pressure second derivative (through finite differences of order 8) and looking for its inflection point. The separation point was instead inferred by the shear stress imposing it as the shear stress zero.
        
        The study was performed by setting \textit{3} different meshes along the $\Theta$ coordinate: 120, 180 and 360 equally spaced cells, while mantaining the $\rho$ grid fixed to 100 cells distributed through a Geometric law with a coefficient of...

        Figure~\ref{fig:drag_independence}, Figure~\ref{fig:pression_ind}, and Figure~\ref{fig:wall_ind} show our results.

        \begin{figure}[!ht]
                \includegraphics[width=\textwidth]{DragCoefficient_Independence.png}
                \centering
                \caption{}
                \label{fig:drag_independence}
        \end{figure}

        \begin{figure}[!ht]
                \includegraphics[width=\textwidth]{Pressure_Independence.png}
                \centering
                \caption{}
                \label{fig:pression_ind}
        \end{figure}

        \begin{figure}[!ht]
                \includegraphics[width=\textwidth]{WallShearStress_Independence.png}
                \centering
                \caption{X-velocity Y-profile per delta-step}
                \label{fig:wall_ind}
        \end{figure}


        According to experimental results separation point s should lie at $$ 82 ^\circ $$, and our plot quite respects this result.

\section{Unsteady-state modelling} \label{sec:URAN}
        In this section we'll make use of our results from URANS simulations. To launch these simulations we re-started from the converged steady-state solutions exploiting the finer grid setting.

        We defined as suitable total simulation time to observe periodicity in the macroscopic flow: 60 s. Then, we performed a sensibility analysis with respect to the time-step of time discretization; we considered three different timesteps, ensuring their value to be much smaller with respect to the total time-scal and bigger than microscopic turbolent time-scales. These were:

        \begin{itemize}
                \item $0.06 s$ ($0.1\%$ of the total simulation time).
                \item $0.10 s$ ($0.17\%$ of the total simulation time).
                \item $0.20 s$ ($0.33\%$ of the total simulation time).
        \end{itemize}
        
        At each time-step we considered as target parameter: forces on the \textit{X} and \textit{Y} axis.

        The behaviour of the Drag and Lift force with the 3 different values imposed can be seen respectively in Figure~\ref{fig:drag} and Figure~\ref{fig:lift}. This figures does not appear to describe correctlythe phenomenon under investigation since the forces under investigation never stabilize even after 100 seconds.

        \begin{figure}[!ht]
                \includegraphics[width=\textwidth]{DragForce.png}
                \centering
                \caption{}
                \label{fig:drag}
        \end{figure}

        \begin{figure}[!ht]
                \includegraphics[width=\textwidth]{LiftForce.png}
                \centering
                \caption{}
                \label{fig:lift}
        \end{figure}

        Regarding the Drag and Lift coefficients evolution over time the $0.06 s$ timestep fails; the numerical results are shown in Figure~\ref{fig:drag_coeff} and  Figure~\ref{fig:lift_coeff}. At first glance despite the lift coefficient oscillations are 0-averaged and drag coefficients oscillations are positively averaged the order of amplitude of the oscillations seems too small. 

        \begin{figure}[!ht]
                \includegraphics[width=\textwidth]{Drag_Coefficient.png}
                \centering
                \caption{}
                \label{fig:drag_coeff}
        \end{figure}
                \begin{figure}[!ht]
                \includegraphics[width=\textwidth]{Lift_Coefficient.png}
                \centering
                \caption{}
                \label{fig:lift_coeff}
        \end{figure}


        Finally both averaged Drag Coefficient and Strouhal number, respectively  $\num{0.6468}$ and $0.1365$ do not match well with expectations of Figure~\ref{fig:dragtrend}, that suggesting more than $1.0$ and Figure~\ref{fig:strouhal}.
        
        To overcome this issues we had to manipulate on timesteps and total time, leading to a final decision of 9000 timesteps and 45 seconds. Probabily previous numbers of timesteps were too big and made some charcteristic patterns disappear. In Figure~\ref{fig:drag_f}, Figure~\ref{fig:drag_vsl1} and Figure~\ref{fig:drag_vsl2} our results can be seen. 
        
               \begin{figure}[!ht]
                \includegraphics[width=\textwidth]{drag_Federica.png}
                \centering
                \caption{}
                \label{fig:drag_f}
        \end{figure}
     
            \begin{figure}[!ht]
                \includegraphics[width=\textwidth]{drag_vs_lift_Federica.png}
                \centering
                \caption{}
                \label{fig:drag_vsl1}
        \end{figure}
     
            \begin{figure}[!ht]
                \includegraphics[width=\textwidth]{drag_vs_lift_2_Federica.png}
                \centering
                \caption{}
                \label{fig:drag_vsl2}
        \end{figure}
        
        With these images we can appreciate how drag and lift frequency is comparable, while their amplitude is of different order. Moreover it is well clear that they're in phase, since subject to the same physical phenomenon. Finally the averaged Cd coefficient is 1.067 while Strouhal number 0.17, which respects better the expectations of Figure~\ref{fig:dragtrend} and of Figure~\ref{fig:strouhal}.
     

\bibliographystyle{abbrv}
\bibliography{main}

\end{document}
