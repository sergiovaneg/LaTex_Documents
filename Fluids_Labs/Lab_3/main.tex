\title{Test Case 2 - Fluids Labs}
\author{
        Sergio M. Vanegas A.\\
                Department of Mathematics\\
        Polimi---Politecnico di Milano\\
        Milano, Italia
}
\date{\today}

\documentclass[12pt]{article}

\usepackage{amsmath}
\usepackage{graphicx}
\usepackage{siunitx}

\begin{document}
\maketitle

\begin{abstract}
        The present case concerns the development of the turbulent flow between two parallel plates. The flow develops from a condition of uniform velocity (rectangular profile) imposed at the inlet boundary, reaching a fully-developed state at a certain distance from the inlet, and it does not change further downstream until the outlet section. Unlike that in the laminar regime (plane Poiseuille flow), the turbulent flow between two parallel plates does not have any analytical solution. In this laboratory, PHOENICS is used to simulate the flow by solving the RANS coupled with the \( k \text{-} \epsilon\) standard turbulence model and the equilibrium wall function of Launder and Spalding. \cite{FL:03}
\end{abstract}

\section{Introduction}

        \subsection{Case Description}
        
        \subsection{Flow Domain}
        
        \subsection{Data of the Problem}
        
        \subsection{Report Structure}

                The remainder of the report is organized as follows: Section~\ref{sec:convergence} addresses the issue of Whole-Field residual convergence, Flow Full-Development and Grid Independence; Section~\ref{sec:qualitative} presents the relevant profiles and simulated scalars for the qualitative assesment of the model physical consistency; Section~\ref{sec:literature} compares the results of the simulation with the theoretical model studied during the lessons; finally, Section~\ref{sec:unstability} studies the behaviour of the CFD simulations when stability conditions (Extremely high Reynolds number/Over-refined wall mesh) are not met.

\section{Convergence of the CFD Solution} \label{sec:convergence}

        a

\section{Qualitative assessment of the physical consistency of the CFD solution} \label{sec:qualitative}

        a

\section{Comparison with literature models} \label{sec:literature}

        a

\section{Optional questions for further individual study} \label{sec:unstability}

        a

\bibliographystyle{abbrv}
\bibliography{main}

\end{document}
