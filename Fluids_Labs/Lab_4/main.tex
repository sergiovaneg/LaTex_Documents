\title{Test Case 3 - Fluids Labs}
\author{
        Sergio M. Vanegas A.\\
        Francesco de Pas\\
                Department of Mathematics\\
        Polimi---Politecnico di Milano\\
        Milano, Italia
}
\date{\today}

\documentclass[12pt]{article}

\usepackage{amsmath}
\usepackage{graphicx}
\usepackage{siunitx}

\begin{document}
\maketitle

\begin{abstract}
        The present case concerns the development of the boundary layer produced by a uniform flow over a flat plate. The flow develops from a condition of uniform velocity \( U_\infty \) imposed at the inlet boundary, growing indefinitely without reaching a fully-developed state. Up to a distance \(x_{\text{lam}}\) from the leading edge such that \( \text{Re}_x = U_\infty x_{\text{lam}} / \nu \approx \num{1E5} \) , the boundary layer remains laminar; beyond a distance \( x_{\text{turb}} \) from the leading edge such that \( \text{Re}_x = U_\infty x_{\text{turb}} / \nu \approx \num{3E6} \) , the boundary layer is fully turbulent; in-between \( x_{\text{lam}} \) and \( x_{\text{turb}} \), transitional boundary layer will occur. In this Laboratory, PHOENICS is used to simulate the development of the laminar boundary layer, whereas the analysis of the turbulent part will be shelved for further individual study. \cite{FL:04}

\end{abstract}

\section{Introduction}
        

        The remainder of the report is organized as follows: Section~\ref{sec:suitability} addreses the general procedure of simulation convergence verification and mesh suitability, whereas Section~\ref{sec:comparison} compares the simulation results against the reference model.

\section{Suitability of the CFD model} \label{sec:suitability}

\section{Comparison against the Blasius solution} \label{sec:comparison}

\bibliographystyle{abbrv}
\bibliography{main}

\end{document}
