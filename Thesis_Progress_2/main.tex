\documentclass{beamer}
\usetheme{CambridgeUS}
\title[Process]{The Koopman operator identification algorithm}
\subtitle{So far}
\institute[Polimi]{Politecnico di Milano}
\author{Sergio Vanegas}
\date{\today}

\usepackage{listings}
\usepackage[framed,numbered,autolinebreaks,useliterate]{mcode/mcode}

\usepackage{xcolor}

\begin{document}

\begin{frame}[plain,noframenumbering]
    \maketitle
\end{frame}

\begin{frame}{Table of Contents}
    \tableofcontents
\end{frame}


\section{Robust Koopman}

\begin{frame}{Proposed regularization}
    We start by following the proposed reformulation of the minimization problem proposed in \textit{Robust tube-based model predictive control with Koopman operators} as follows:

    \begin{equation*}
        \overline{\mathcal{K}}_N =
        \min_{\mathcal{K} \in \mathbb{R}^{\tilde{N} \times N}}
        \sum_{k=1}^K \left[
            \left|\left|
                \mathbf{g}\left(\mathbf{y}\left[k\right]\right) - 
                \mathcal{K}
                \begin{pmatrix}
                \mathbf{g}\left(\mathbf{x}\left[k\right]\right)
                \\
                \mathbf{u}\left[k\right]
                \end{pmatrix}
            \right|\right|^2
            {
                \color{red}
                + \alpha\left|\left|\mathcal{K}\right|\right|_F^2
            }
        \right],
    \end{equation*}

    where

    \begin{itemize}
        \item $\mathbf{g}$ is the observable vector,
        \item $K$ is the total number of snapshots,
        \item $\left(\mathbf{x},\mathbf{y}\right)$ are the snapshot pairs,
        \item $\mathbf{u}$ is the input vector,
        \item $\left|\left|\cdot\right|\right|_F$ is the element-wise 2-norm (Frobenius norm).
    \end{itemize}
\end{frame}

\end{document}