\documentclass{article}

\usepackage{amsmath}
\usepackage{amssymb}

\usepackage{comment}

\usepackage{cleveref}

\bibliographystyle{ieeetr}

\begin{document}

\section{The Abstract Koopman Operator}

    Koopman operators consist on a dimension lifting technique that allows us to model a Finite Dimensional Nonlinear Dynamical System through an Infinite Dimensional Linear Dynamical System.
    
    Koopman operators provide a convenient framework where traditional, more mature and optimized Linear System Identification and Control tools can be applied. In contrast with conventional linearization methods that depend on local gradients, the Koopman operator provides an exact description (at least in theory) of the System's Dynamics and, more importantly from the Applied Controls point of view, it is easily adaptable to a fully data-driven pipeline, provided we have access to the internal states of the System of interest (gray-box framework).

    We start by stating the abstract problem, partially borrowing the definition from Mezić et Al.\cite{Koopman_Basics}\cite{Applied_Koopmanism}. Consider a continuous-time dynamical system such as the one given in \cref{eq:CT_Dyn} on the state space $M$, where $\mathbf{x}$ is a coordinate vector of the state, and $\textbf{F}$ is (assumed to be) a non-linear vector-valued smooth function of the same dimension as its argument.

    \begin{equation} \label{eq:CT_Dyn}
        \mathbf{\dot{x}} = \mathbf{F}\left( \mathbf{x} \right) , \mathbf{x} \in M
    \end{equation}
    
    Let $S^t(\mathbf{x}_0)$ denote the position the position at time $t$ of the trajectory of the system in \cref{eq:CT_Dyn} that starts at $t=0$ from the state $\mathbf{x}_0$, and let $g:M \mapsto \mathbb{C}$ an \textit{arbitrary} (at least for now), complex-valued function. We call $g$ an observable of the system in \cref{eq:CT_Dyn}, whose value observed over a trajectory starting from $\mathbf{x}_0$ at $t=0$ changes with time according to the flow, as described in \cref{eq:obs_dyn}.

    \begin{equation} \label{eq:obs_dyn}
        g\left(t,\mathbf{x}_0\right) = g \left(\mathbf{S}^t\left(\mathbf{x}_0\right)\right)
    \end{equation}
    
    The space of all observables such as $g$ is a linear functional space, and thus we can define a family of linear operators $U^t:\mathbb{C} \mapsto \mathbb{C}$ with $t \in \left[0,\infty\right)$ through \cref{eq:Koopman_CT}.

    \begin{equation} \label{eq:Koopman_CT}
        U^t g\left(\mathbf{x}_0\right) = g \left(S^t \left(\mathbf{x}_0\right)\right)
    \end{equation}

    Thus, for a fixed $t$, $U^t$ maps the vector-valued observable $g(\mathbf{x}_0)$ into $g\left(t,\mathbf{x}_0\right)$. We will call the family of all operators $U^t$, indexed by time $t$, the \textbf{Koopman Semi-group} given by the generator $U$ of the continuous-time system in \cref{eq:CT_Dyn}.

    In Discrete Time, the definition is even simpler. Let \cref{eq:DT_Dyn} describe a discrete-time dynamical system with $\mathbf{z} \in M$ and $\mathbf{F}:M \mapsto M$. Then, the associated Koopman operator $U$ is defined by \cref{eq:Koopman_DT}.

    \begin{equation} \label{eq:DT_Dyn}
        \textbf{z}' = \textbf{F}\left(\textbf{z}\right)
    \end{equation}

    \begin{equation} \label{eq:Koopman_DT}
        U g\left(\mathbf{z}\right) = g \circ \mathbf{F} \left(\mathbf{z}\right)
    \end{equation}

    At this point, it is worth noting that the discrete-time Koopman operator is also linear; i.e., $U\left(c_1 \mathbf{f}_1\left(\mathbf{z}\right) + c_2 \mathbf{f}_2\left(\mathbf{z}\right)\right) =
    c_1 \mathbf{f}_1\left(\mathbf{T}\left(\mathbf{z}\right)\right) + c_2 \mathbf{f}_2\left(\mathbf{T}\left(\mathbf{z}\right)\right) =
    c_1 U\mathbf{f}_1\left(\mathbf{z}\right) + c_2 U\mathbf{f}_2\left(\mathbf{z}\right)$. This linearity allows us to derive \cref{eq:DT_Koopman_Concat}, which implies that the dynamics of the observables of a system over a trajectory $S^k\left(\mathbf{z}_0\right)$ can be entirely described by the consecutive application of the discrete-time Koopman operator without having to "observe" the system twice for a single trajectory.

    \begin{equation} \label{eq:DT_Koopman_Concat}
        g \circ \mathbf{F}\left(\mathbf{F}\left(\mathbf{z}_0\right)\right) = U g\left(\mathbf{F}\left(\mathbf{z}_0\right)\right) = U g \circ \mathbf{F} \left(\mathbf{z}_0\right) = U^2 g\left(\mathbf{z}_0\right) , \mathbf{z}_0 \in M
    \end{equation}
    
\begin{comment}
    By taking advantage of this linearity, we can now make use of eigenfunction decomposition in order to obtain further information from the defined operator. We call $\phi : M \mapsto \mathbb{C}$ an eigenfunction of the Koopman operator $U$, associated to an eigenvalue $\lambda \in \mathbb{C}$.
\end{comment}

\section{The Koopman Operator for Real System\\Identification}

    After properly defining the abstract the Koopman operator, we now want to stablish a practical framework in order to make use of its properties for System Identification and Control applications. We start by redefining $M$ as a real vector-space, and we follow the numerical scheme proposed by Mauroy-Goncalves\cite{Goncalves} as a data-driven Identification method for dynamical systems.
    
    We consider the Real Dynamical System described in \cref{eq:CT_Dyn_Real}, where $n$ is the dimension of the system. 

    \begin{equation} \label{eq:CT_Dyn_Real}
        \dot{\mathbf{x}} = \mathbf{F}\left(\mathbf{x}\right), \mathbf{x} \in \mathbb{R}^n
    \end{equation}

    The vector field $\mathbf{F}$ is w.l.o.g. of the form described in \cref{eq:F_Decomp}, where the $k$ vectors $\mathbf{w}_i = \left( w_i^1 \cdots w_i^n \right)^T \in \mathbb{C}^n$ are unknown coefficients to be identified, whereas the $k$ library functions $h_i : \mathbb{R}^n \mapsto \mathbb{C}$ are assumed to be known; note that some coefficients might be zero.

    \begin{equation} \label{eq:F_Decomp}
        \mathbf{F}(\mathbf{x}) = \sum_{i = 1}^{k} \mathbf{w}_i h_i \left( \mathbf{x} \right)
    \end{equation}

    At this point, a choice has to be made: so far, we have treated the collection of observables as an infinite collection with all possible functions mapping the original state space into a finite dimensional 

    \bibliography{main.bib}

\end{document}