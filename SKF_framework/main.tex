\documentclass{beamer}
\usetheme{Antibes}
\usecolortheme{beaver}

\title{Bearing Condition Monitoring}
\subtitle{Quadro Teorico}
\author{Sergio Vanegas}
\institute{Modelway S.r.l.}
\date{\today}

\begin{document}

\frame{\titlepage}

\begin{frame}{Table of Contents}
    \tableofcontents
\end{frame}

\section{Introduzione}

\begin{frame}{Oggetto dell'intervento}
    \begin{itemize}
        \item SKF: uno dei leader mondiali nel settore bearing.
        \item La failure/anomaly detection dei cuscinetti risulta importante per:
        \begin{itemize}
            \item Rilevamento di situazioni pericolose.
            \item implementazione della manutenzione programata e preventiva.
        \end{itemize}
        \item Goal: fornire a SKF un tool software che rilevi automaticamente la presenza di failure od anomalie sul cuscinetto.
    \end{itemize}
\end{frame}

\section{Stato dell'arte}

\begin{frame}{Analisi ad Elementi Finiti}
    
\end{frame}

\end{document}