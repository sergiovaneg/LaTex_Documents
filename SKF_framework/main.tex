\documentclass{beamer}

\usepackage{biblatex}
\bibliography{bibliography}
\AtBeginBibliography{\small}
\usepackage{hyperref}
\usepackage{siunitx}

\usetheme{Antibes}
\usecolortheme{beaver}

\title{Bearing Condition Monitoring}
\subtitle{Quadro Teorico}
\author{Sergio Vanegas}
\institute{Modelway S.r.l.}
\date{\today}

\begin{document}

\frame{\titlepage}

\begin{frame}{Table of Contents}
    \tableofcontents
\end{frame}

\section{Introduzione}

\begin{frame}{Oggetto dell'intervento}
    \begin{itemize}
        \item SKF: uno dei leader mondiali nel settore bearing.
        \item La failure/anomaly detection dei cuscinetti risulta importante per:
        \begin{itemize}
            \item Rilevamento di situazioni pericolose.
            \item Implementazione della manutenzione programmata e preventiva.
        \end{itemize}
        \item Goal: fornire a SKF un tool software che rilevi automaticamente la presenza di failure o anomalie sul cuscinetto.
    \end{itemize}

    \begin{figure}
        \centering
        \includegraphics[height=0.3\textheight]{Figures/diagram.jpg}
    \end{figure}
\end{frame}

\section{Pre-analisi}

\begin{frame}[allowframebreaks]{Analisi ad Elementi Finiti}
    Forti stress multiassiali sui contatti delle piste interne ed esterne governati dalla teoria di Hertz.

    \begin{figure}
        \centering
        \includegraphics[width=\textwidth]{Figures/FE_Life_Curve.png}
        \caption{Relazione tra la vita del cuscinetto e lo stress medio di contatto}
        \label{fig:FE_Life_Curve}
    \end{figure}

    \framebreak

    \begin{itemize}
        \item Modellazione FE (simmetrica) con imposizione di duty cycle\cite{raju2018bearing}.
        \item I cuscinetti si guastano tipicamente a causa della Rolling Contact Fatigue (sotto la superficie).
        \item Buona correlazione tra ``FE contact stress'' e la vita utile dei cuscinetti.
        \item Modello adattato modificando:
        \begin{itemize}
            \item La distribuzione del duty cycle.
            \item I fattori di rugosità superficiale/microgeometria.
        \end{itemize}
    \end{itemize}
\end{frame}

\begin{frame}[allowframebreaks]{Analisi del pre-carico}
    Sulla base del modello matematico e dello spettro di carico, viene calcolata e stimata la durata dell'unità cuscinetto del mozzo della ruota\cite{niu2014life}.

    \begin{figure}
        \centering
        \includegraphics[width=0.5\textwidth]{Figures/Preload_Life_Estimation.png}
        \caption{Relazione tra pre-carico e vita utile}
        \label{fig:Preload_Life_Curve}
    \end{figure}
    
    Dall'altra parte, la frequenza naturale del cuscinetto aumenta con la forza di pre-carico (questa tendenza tende a rallentare).

    \begin{figure}
        \centering
        \includegraphics[width=0.4\textwidth]{Figures/Preload_NatFreq.png}
        \caption{Relazione tra pre-carico e frequenza naturale}
        \label{fig:Preload_Freq}
    \end{figure}

    Questo significa che si può stabilire una relazione tra la frequenza naturale del cuscinetto e la sua vita utile.
\end{frame}

\section{Monitoraggio}

\begin{frame}[allowframebreaks]{Frequenze delle vibrazioni}
    Parametri del cuscinetto:
    \begin{itemize}
        \item $d$: diametro delle palline
        \item $D$: diametro del cuscinetto
        \item $f_r$: frequenza di giro del fusto
        \item $n$: numero di palline
        \item $\phi$: angolo di contatto del cuscinetto
    \end{itemize}

    Relevant frequencies:
    \begin{itemize}
        \item Ballpass frequency - Canale esterno (BPFO): $$ BPFO = \frac{n f_r}{2} \left( 1 - \frac{d}{D} \cos \phi \right) $$ (A lot harder to detect; requires Kurtosis-informed BP filtering)
        \item Ballpass frequency - Canale interno (BPFI): $$ BPFI = \frac{n f_r}{2} \left( 1 + \frac{d}{D} \cos \phi \right) $$
        \item Fundamental train frequency (FTF), o velocità della gabbia: $$ FTF = \frac{f_r}{2} \left( 1 - \frac{d}{D} \cos \phi \right) $$
        \item Ball spin frequency (BSF): $$ BSF = \frac{D}{2 d} \left[ 1 - \left( \frac{d}{D} \cos \phi \right)^2 \right] $$
    \end{itemize}

    \framebreak
    
    Tecniche d'analisi:
    \begin{itemize}
        \item Decomposizione del(l'inviluppo del) segnale\cite{tang2019fault}:
        \begin{itemize}
            \item Frequenze del cuscinetto.
            \item Frequenze dei difetti.
            \item Frequenze del rumore.
        \end{itemize}
        \item Analisi statistica dello spettro\cite{xue2022fault}:
        \begin{itemize}
            \item Decomposizione tramite \textit{Resonance-based Signal Sparse Decomposition} (RSSD).
            \begin{itemize}
                \item Risonanza trovata utilizzando \textit{Tunable Q-factor Wavelet Transform} (TQWT).
            \end{itemize}
            \item Metodo di estrazione della frequenza di difetto basato su \textit{Ratio of Smoothness and Kurtosis} (RSK): $$RSK = \exp^{x_1/N} \cdot \sigma^4 \left( \mathbb{E}\left\{ (x_2 - \mu)^4 \right\} \cdot x_1 / N \right)^{-1}$$
        \end{itemize}

        \begin{figure}
            \centering
            \includegraphics[height=0.7\textheight]{Figures/RSK_RSSD.png}
            \caption{Algoritmo dell'analisi statistica delle frequenze}
            \label{fig:RSK_RSSD}
        \end{figure}

        \item Analisi diretto dello spettro basata su STFT\cite{rubio2012experimental}.
    \end{itemize}
    
    \begin{figure}
        \centering
        \includegraphics[width=0.8\textwidth]{Figures/STFT.png}
        \caption{STFT delle vibrazioni (cuscinetto buono/danneggiato)}
        \label{fig:STFT}
    \end{figure}
\end{frame}

\begin{frame}{Carica multiassiale della ruota}
    \small
    È possibile stabilire una legge per determinare quando sostituire i cuscinetti in funzione dell'equilibrio delle forze\cite{zhao2021service}.

    \begin{figure}
        \centering
        \includegraphics[width=0.3\textwidth]{Figures/Force_Diagram.png}
        \caption{Schema delle forze sul cuscinetto del mozzo ruota}
        \label{fig:Forces_Bearing}
    \end{figure}

    Contando il ciclo di distribuzione combinato dei carichi multiassiali e classificando i tipi di carica, la vita utile del cuscinetto è calcolata in base al \textbf{carico di contatto circonferenziale} del cuscinetto.
\end{frame}

\begin{frame}{Rumore acustico}
    Analisi del segnale basata sull'autocorrelazione in tempo discreto.

    \begin{figure}
        \centering
        \includegraphics[width=0.6\textwidth]{Figures/Acoustic.png}
        \caption{Autocorrelazione del rumore acustico ($f_s = \SI{20}{\kilo \hertz}$)}
        \label{fig:acoustic}
    \end{figure}

    Anche se l'ambiente non è ideale per l'analisi del rumore acustico, la caratterizzazione delle frequenze del cuscinetto e dei difetti può essere usata per calibrare un filtro Band-Pass.

\end{frame}

\section{Conclusioni}

\begin{frame}{Conclusioni}
    \begin{itemize}
        \item Tra tutte le tecniche presentate, le più promettenti sono:
        \begin{itemize}
            \item L'analisi della grandezza delle frequenze caratteristiche dall'inviluppo del segnale;
            \item L'analisi statistica dello spettro del segnale.
        \end{itemize}
        \item Sebbene non è possibile implementare l'analisi agli elementi finiti in tempo reale, è possibile utilizzare le informazioni fornite da questa tecnica per mettere a punto il modello di previsione per un disegno specifico di WHB.
        \item Nel caso in cui sia possibile aggiungere altri sensori al sistema di monitoraggio, utilizzare un microfono potrebbe essere un complemento non invasivo alle informazioni fornite dall'accelerometro.
    \end{itemize}
\end{frame}

\section*{Bibliografia}

\begin{frame}[allowframebreaks]{Bibliografia}
    \printbibliography
\end{frame}

\end{document}