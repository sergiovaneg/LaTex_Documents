% A LaTeX template for ARTICLE version of the MSc Thesis submissions to 
% Politecnico di Milano (PoliMi) - School of Industrial and Information Engineering
%
% S. Bonetti, A. Gruttadauria, G. Mescolini, A. Zingaro
% e-mail: template-tesi-ingind@polimi.it
%
% Last Revision: October 2021
%
% Copyright 2021 Politecnico di Milano, Italy. Inc. NC-BY

\documentclass[11pt,a4paper]{article} 

%------------------------------------------------------------------------------
%	REQUIRED PACKAGES AND  CONFIGURATIONS
%------------------------------------------------------------------------------
% PACKAGES FOR TITLES
\usepackage{titlesec}
\usepackage{color}

% PACKAGES FOR LANGUAGE AND FONT
\usepackage[utf8]{inputenc}
\usepackage[english]{babel}
\usepackage[T1]{fontenc} % Font encoding

% PACKAGES FOR IMAGES
\usepackage{graphicx}
\graphicspath{{Images/}}
\usepackage{eso-pic} % For the background picture on the title page
\usepackage{subfig} % Numbered and caption subfigures using \subfloat
\usepackage{caption} % Coloured captions
\usepackage{transparent}

% STANDARD MATH PACKAGES
\usepackage{amsmath}
\usepackage{amsthm}
\usepackage{bm}
\usepackage[overload]{empheq}  % For braced-style systems of equations

% PACKAGES FOR TABLES
\usepackage{tabularx}
\usepackage{longtable} % tables that can span several pages
\usepackage{colortbl}

% PACKAGES FOR ALGORITHMS (PSEUDO-CODE)
\usepackage{algorithm}
\usepackage{algorithmic}

% PACKAGES FOR REFERENCES & BIBLIOGRAPHY
\usepackage[colorlinks=true,linkcolor=black,anchorcolor=black,citecolor=black,filecolor=black,menucolor=black,runcolor=black,urlcolor=black]{hyperref} % Adds clickable links at references
\usepackage{cleveref}
\usepackage[square, numbers, sort&compress]{natbib} % Square brackets, citing references with numbers, citations sorted by appearance in the text and compressed
\bibliographystyle{plain} % You may use a different style adapted to your field

% PACKAGES FOR THE APPENDIX
\usepackage{appendix}

% PACKAGES FOR ITEMIZE & ENUMERATES 
\usepackage{enumitem}

% OTHER PACKAGES
\usepackage{amsthm,thmtools,xcolor} % Coloured "Theorem"
\usepackage{comment} % Comment part of code
\usepackage{fancyhdr} % Fancy headers and footers
\usepackage{lipsum} % Insert dummy text
\usepackage{tcolorbox} % Create coloured boxes (e.g. the one for the key-words)

%-------------------------------------------------------------------------
%	NEW COMMANDS DEFINED
%-------------------------------------------------------------------------
% EXAMPLES OF NEW COMMANDS -> here you see how to define new commands
\newcommand{\bea}{\begin{eqnarray}} % Shortcut for equation arrays
\newcommand{\eea}{\end{eqnarray}}
\newcommand{\e}[1]{\times 10^{#1}}  % Powers of 10 notation
\newcommand{\mathbbm}[1]{\text{\usefont{U}{bbm}{m}{n}#1}} % From mathbbm.sty
\newcommand{\pdev}[2]{\frac{\partial#1}{\partial#2}}
% NB: you can also override some existing commands with the keyword \renewcommand

%----------------------------------------------------------------------------
%	ADD YOUR PACKAGES (be careful of package interaction)
%----------------------------------------------------------------------------


%----------------------------------------------------------------------------
%	ADD YOUR DEFINITIONS AND COMMANDS (be careful of existing commands)
%----------------------------------------------------------------------------


% Do not change Configuration_files/config.tex file unless you really know what you are doing. 
% This file ends the configuration procedures (e.g. customizing commands, definition of new commands)
\input{Configuration_files/config}

% Insert here the info that will be displayed into your Title page 
% -> title of your work
\renewcommand{\title}{Fast Marching Method for the Eikonal model of the Cardiac Electrophysiology}
% -> author name and surname
\renewcommand{\author}{Donato Cerrone, Francesco de Pas, Sergio Vanegas}
% -> MSc course
\newcommand{\course}{Numerical Analysis for Partial Differential Equations\\Mathematical Engineering - Ingegneria Matematica}
% -> advisor name and surname
\newcommand{\advisor}{Prof. Name Surname}
% IF AND ONLY IF you need to modify the co-supervisors you also have to modify the file Configuration_files/title_page.tex (ONLY where it is marked)
\newcommand{\firstcoadvisor}{Name Surname} % insert if any otherwise comment
\newcommand{\secondcoadvisor}{Name Surname} % insert if any otherwise comment
% -> author ID
\newcommand{\ID}{10615424, 10579839, 10788735}
% -> academic year
\newcommand{\YEAR}{2020-2021}
% -> abstract (only in English)
\renewcommand{\abstract}{
    Here goes the Abstract in English of your thesis (in article format) followed by a list of keywords. The Abstract is a concise summary of the content of the thesis (single page of text) and a guide to the most important contributions included in your thesis. The Abstract is the very last thing you write. It should be a self-contained text and should be clear to someone who hasn't (yet) read the whole manuscript. The Abstract should contain the answers to the main research questions that have been addressed in your thesis. It needs to summarize the motivations and the adopted approach as well as the findings of your work and their relevance and impact. The Abstract is the part appearing in the record of your thesis inside POLITesi, the Digital Archive of PhD and Master Theses (Laurea Magistrale) of Politecnico di Milano. The Abstract will be followed by a list of four to six keywords. Keywords are a tool to help indexers and search engines to find relevant documents. To be relevant and effective, keywords must be chosen carefully. They should represent the content of your work and be specific to your field or sub-field. Keywords may be a single word or two to four words.
}

% -> key-words (only in English)
\newcommand{\keywords}{here, the keywords, of your thesis}

%-------------------------------------------------------------------------
%	BEGIN OF YOUR DOCUMENT
%-------------------------------------------------------------------------
\begin{document}

%-----------------------------------------------------------------------------
% TITLE PAGE
%-----------------------------------------------------------------------------
% Do not change Configuration_files/TitlePage.tex (Modify it IF AND ONLY IF you need to add or delete the Co-advisors)
% This file creates the Title Page of the document
\input{Configuration_files/title_page}

%%%%%%%%%%%%%%%%%%%%%%%%%%%%%%
%%     THESIS MAIN TEXT     %%
%%%%%%%%%%%%%%%%%%%%%%%%%%%%%%

%-----------------------------------------------------------------------------
% INTRODUCTION
%-----------------------------------------------------------------------------
\section{Introduction} \label{sec:introduction}

    One of the most challenging problems nowadays consists in describing biological processes through a formal mathematical framework. Fitting accurate numerical models upon relevant medical phenomena often requires a huge number of experimental data and getting them revealed to be troublesome. As a consequence adjusting complete models to specific medical cases is often hard and reduced ones are of great interest.

    In this paper, we consider as physiological process the propagation of the electrical signal through the myocardium. This process is well described by the Bidomain Model whose equations derive from a circuital representation of the heart's muscle cells. For computational reason, two simplification are usually adopted: the Monodomain model and the Eikonal model; in particular, we will focus on the latter whose general formulation, neglegting boundary conditions, reads
    
    $$ F\sqrt{\nabla T ^t D \nabla T} $$

    This is a function of $T$, where $T(\textbf{x})$ is the arrival time at which the electric signal reaches the position \textbf{x}, $F$ is the speed term, and $D$ the conductivity stress tensor.

    The clinical usage of these results forces to find an accurate and fast way to solve this problem. Our analysis relies on the so called \textit{Fast Marching Method} (FMM), which aims at satisfying both these requirements. The core idea of FMM consists in exploiting the physical properties of the signal spread inside the myocardium, computing the arrival time of the peak value isocontour at each point.

    Our work mainly aims at implementing this method in the LIFEX environment, highlighting its advantages and drawbacks with respect to the already implemented solution of the Eikonal Model which relies on FEM.

%-----------------------------------------------------------------------------
% EQUATIONS
%-----------------------------------------------------------------------------
\section{Equations}

    \label{sec:eqs}
    This section gives some examples of writing mathematical equations in your thesis.

    Maxwell's equations read:
    \begin{subequations}
        \label{eq:maxwell}
        \begin{align}[left=\empheqlbrace]
            \nabla\cdot \bm{D} & = \rho, \label{eq:maxwell1} \\
            \nabla \times \bm{E} +  \frac{\partial \bm{B}}{\partial t} & = \bm{0}, \label{eq:maxwell2} \\
            \nabla\cdot \bm{B} & = 0, \label{eq:maxwell3} \\
            \nabla \times \bm{H} - \frac{\partial \bm{D}}{\partial t} &= \bm{J}. \label{eq:maxwell4}
        \end{align}
    \end{subequations}

    Equation~\eqref{eq:maxwell} is automatically labeled by \texttt{cleveref}, as well as Equation~\eqref{eq:maxwell1} and Equation~\eqref{eq:maxwell3}. Thanks to the \verb|cleveref| package, there is no need to use \verb|\eqref|. Equations have to be numbered only if they are referenced in the text.

    Equations~\eqref{eq:maxwell_multilabels1}, \eqref{eq:maxwell_multilabels2}, \eqref{eq:maxwell_multilabels3}, and \eqref{eq:maxwell_multilabels4} show again Maxwell's equations without brace:

    \begin{align}
        \nabla\cdot \bm{D} & = \rho, \label{eq:maxwell_multilabels1} \\
        \nabla \times \bm{E} +  \frac{\partial \bm{B}}{\partial t} &= \bm{0}, \label{eq:maxwell_multilabels2} \\
        \nabla\cdot \bm{B} & = 0, \label{eq:maxwell_multilabels3} \\
        \nabla \times \bm{H} - \frac{\partial \bm{D}}{\partial t} &= \bm{J} \label{eq:maxwell_multilabels4}.
    \end{align}

    Equation~\eqref{eq:maxwell_singlelabel} is the same as before,
    but with just one label:

    \begin{equation}
        \label{eq:maxwell_singlelabel}
        \left\{
        \begin{aligned}
            \nabla\cdot \bm{D} & = \rho, \\
            \nabla \times \bm{E} +  \frac{\partial \bm{B}}{\partial t} &= \bm{0},\\
            \nabla\cdot \bm{B} & = 0, \\
            \nabla \times \bm{H} - \frac{\partial \bm{D}}{\partial t} &= \bm{J}.
        \end{aligned}
        \right.
    \end{equation}

%-----------------------------------------------------------------------------
% FIGURES, TABLES AND ALGORITHMS
%-----------------------------------------------------------------------------
\section{Figures, Tables and Algorithms}

    Figures, Tables and Algorithms have to contain a Caption that describes their content, and have to be properly referred in the text.

    \subsection{Figures} \label{subsec:figures}

        For including pictures in your text you can use \texttt{TikZ} for high-quality hand-made figures \cite{tikz}, or just include them with the command

        \begin{verbatim}
            \includegraphics[options]{filename.xxx}
        \end{verbatim}

        Here xxx is the correct format, e.g.  \verb|.png|, \verb|.jpg|, \verb|.eps|, \dots.

        \begin{figure}[H]
            \centering
            \includegraphics[width=0.3\textwidth]{logo_polimi_scritta.eps}
            \caption{Caption of the Figure.}
            \label{fig:quadtree}
        \end{figure}

        Thanks to the \texttt{\textbackslash subfloat} command, a single figure, such as Figure~\ref{fig:quadtree}, can contain multiple sub-figures with their own caption and label, e.g. Figure~\ref{fig:polimi_logo1} and Figure~\ref{fig:polimi_logo2}. 

        \begin{figure}[H]
            \centering
            \subfloat[One PoliMi logo.\label{fig:polimi_logo1}]{
                \includegraphics[scale=0.5]{Images/logo_polimi_scritta.eps}
            }
            \quad
            \subfloat[Another one PoliMi logo.\label{fig:polimi_logo2}]{
                \includegraphics[scale=0.5]{Images/logo_polimi_scritta2.eps}
            }
            \caption[]{Caption of the Figure.}
            \label{fig:quadtree2}
        \end{figure}

    \subsection{Tables} \label{subsec:tables}

        Within the environments \texttt{table} and  \texttt{tabular} you can create very fancy tables as the one shown in Table~\ref{table:example}.

        \begin{table}[H]
            \caption*{\textbf{Example of Table (optional)}}
            \centering 
            \begin{tabular}{|p{3em} c c c |}
            \hline
            \rowcolor{bluePoli!40}
            & \textbf{column1} & \textbf{column2} & \textbf{column3} \T\B \\
            \hline \hline
            \textbf{row1} & 1 & 2 & 3 \T\B \\
            \textbf{row2} & $\alpha$ & $\beta$ & $\gamma$ \T\B\\
            \textbf{row3} & alpha & beta & gamma \B\\
            \hline
            \end{tabular}
            \\[10pt]
            \caption{Caption of the Table.}
            \label{table:example}
        \end{table}

        You can also consider to highlight selected columns or rows in order to make tables more readable. Moreover, with the use of \texttt{table*} and the option \texttt{bp} it is possible to align them at the bottom of the page. One example is presented in Table~\ref{table:exampleC}. 

        \begin{table*}[bp]
        \centering 
            \begin{tabular}{|p{3em} | c | c | c | c | c | c|}
            \hline
        %    \rowcolor{bluePoli!40}
            & \textbf{column1} & \textbf{column2} & \textbf{column3} & \textbf{column4} & \textbf{column5} & \textbf{column6} \T\B \\
            \hline \hline
            \textbf{row1} & 1 & 2 & 3 & 4 & 5 & 6 \T\B\\
            \textbf{row2} & a & b & c & d & e & f \T\B\\
            \textbf{row3} & $\alpha$ & $\beta$ & $\gamma$ & $\delta$ & $\phi$ & $\omega$ \T\B\\
            \textbf{row4} & alpha & beta & gamma & delta & phi & omega \B\\
            \hline
            \end{tabular}
            \\[10pt]
            \caption{Highlighting the columns}
            \label{table:exampleC}
        \end{table*}

    \subsection{Algorithms} \label{subsec:algorithms}

        Pseudo-algorithms can be written in \LaTeX{} with the \texttt{algorithm} and \texttt{algorithmic} packages. An example is shown in Algorithm~\ref{alg:var}.

        \begin{algorithm}[H]
            \label{alg:example}
            \caption{Name of the Algorithm}
            \label{alg:var}
            \label{protocol1}
            \begin{algorithmic}[1]
                \STATE Initial instructions
                \FOR{$for-condition$}
                    \STATE{Some instructions}
                    \IF{$if-condition$}
                        \STATE{Some other instructions}
                    \ENDIF
                \ENDFOR
                \WHILE{$while-condition$}
                    \STATE{Some further instructions}
                \ENDWHILE
                \STATE Final instructions
            \end{algorithmic}
        \end{algorithm} 

\section{Some further useful suggestions}

    Theorems have to be formatted as follows:

    \begin{theorem} \label{a_theorem}
        Write here your theorem. 
    \end{theorem}
    \textit{Proof.} If useful you can report here the proof.
    \vspace{0.3cm} % Insert vertical space

    Propositions have to be formatted as follows:

    \begin{proposition}
        Write here your proposition.
    \end{proposition}
    \vspace{0.3cm} % Insert vertical space

    How to insert itemized lists:

    \begin{itemize}
        \item first item;
        \item second item.
    \end{itemize}

    How to write numbered lists:

    \begin{enumerate}
        \item first item;
        \item second item.
    \end{enumerate}

\section{Use of copyrighted material}

    Each student is responsible for obtaining copyright permissions, if necessary, to include published material in the thesis. This applies typically to third-party material published by someone else.

\section{Plagiarism}

    You have to be sure to respect the rules on Copyright and avoid an involuntary plagiarism. It is allowed to take other persons' ideas only if the author and his original work are clearly mentioned. As stated in the Code of Ethics and Conduct, Politecnico di Milano \textit{promotes the integrity of research, condemns manipulation and the infringement of intellectual property}, and gives opportunity to all those who carry out research activities to have an adequate training on ethical conduct and integrity while doing research. To be sure to respect the copyright rules, read the guides on Copyright legislation and citation styles available at:

    \begin{verbatim}
        https://www.biblio.polimi.it/en/tools/courses-and-tutorials
    \end{verbatim}

    You can also attend the courses which are periodically organized on "Bibliographic citations and bibliography management".

%-----------------------------------------------------------------------------
% CONCLUSION
%-----------------------------------------------------------------------------
\section{Conclusions}

    \color{black}
    A final section containing the main conclusions of your research/study
    and possible future developments of your work have to be inserted in the section ``Conclusions''.

\section{Bibliography and citations}

    Your thesis must contain a suitable Bibliography which lists all the sources consulted on developing the work. The list of references is placed at the end of the manuscript after the chapter containing the conclusions. It is suggested to use the BibTeX package and save the bibliographic references in the file \verb|bibliography.bib|. This is indeed a database containing all the information about the references. To cite in your manuscript, use the \verb|\cite{}| command as follows:

    \textit{Here is how you cite bibliography entries: \cite{knuth74}, or multiple ones at once: \cite{knuth92,lamport94}}.

    The bibliography and list of references are generated automatically by running BibTeX \cite{bibtex}.

%-----------------------------------------------------------------------------
% BIBLIOGRAPHY
%-----------------------------------------------------------------------------
\bibliography{bibliography.bib}

\appendix
\section{Appendix A}

    If you need to include an appendix to support the research in your thesis, you can place it at the end of the manuscript. An appendix contains supplementary material (figures, tables, data, codes, mathematical proofs, surveys, \dots) which supplement the main results contained in the previous sections.

\section{Appendix B}

    It may be necessary to include another appendix to better organize the presentation of supplementary material.

%%%%%%%%%%%%%%%%%%%%%%%%%%%%%%%%%%%%%%%%%%%%%%%%%%%%%%%%%%%%%%
%%     ABSTRACT IN ITALIAN LANGUAGE AND ACKNOWLEDGMENTS     %%
%%%%%%%%%%%%%%%%%%%%%%%%%%%%%%%%%%%%%%%%%%%%%%%%%%%%%%%%%%%%%%
\cleardoublepage

%-----------------------------------------------------------------------------
% SOMMARIO
%-----------------------------------------------------------------------------
\section*{Abstract in lingua italiana}
    Qui va l'Abstract in lingua italiana della tesi seguito dalla lista di parole chiave.
    
    \vspace{15pt}
    \begin{tcolorbox}[arc=0pt, boxrule=0pt, colback=bluePoli!60, width=\textwidth, colupper=white]
        \textbf{Parole chiave:} qui, le parole chiave, della tesi, in italiano 
    \end{tcolorbox}

%-----------------------------------------------------------------------------
% ACKNOWLEDGEMENTS
%-----------------------------------------------------------------------------
\section*{Acknowledgements}
    Here you might want to acknowledge someone.

%-------------------------------------------------------------------------
%	END OF YOUR DOCUMENT
%-------------------------------------------------------------------------
\end{document}
