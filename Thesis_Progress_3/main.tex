\documentclass{beamer}
\usetheme{CambridgeUS}
\title[Process]{The Koopman operator identification algorithm}
\subtitle{Nonlinear MPC over the Van Der Pol oscillator}
\institute[Polimi]{Politecnico di Milano}
\author{Sergio Vanegas}
\date{\today}

\usepackage{listings}
\usepackage[framed,numbered,autolinebreaks,useliterate]{mcode/mcode}

\usepackage{caption}
\usepackage{subcaption}

\usepackage{siunitx}


\begin{document}

\begin{frame}[plain,noframenumbering]
    \maketitle
\end{frame}

\begin{frame}{Table of Contents}
    \tableofcontents
\end{frame}

\section{Introduction}

\begin{frame}{Introduction}
    In this presentation, we evaluate the performance of the Koopman algorithm developed through the past weeks when applied to a nonlinear MPC controller in closed-loop.

    We start by designing the reference controller, using a Forward-Euler approximation of the system as the internal predictor inside the MATLAB Toolbox for nonlinear MPC, and tune the weights so as to manipulate the value of the second state using an external input.

    We then use the aforementioned controller to drive the oscillator in Simulink (using the automatic integration methods to get the states out of the continuous-time dynamic system definition), using a frequency sweep as reference and adding noise both to the (theoretically) non-tracked reference and the measured outputs.

    Finally, we extract the Koopman matrices from the Simulink data and evaluate its performance against the original controller over different benchmark references, including the 0 reference.
\end{frame}


\section{Conclusions}

\begin{frame}{Conclusions}
    \begin{itemize}
        \item Lower sampling periods have shown to yield a better prediction horizon, since in practice it is bounded by the variations in shape between different trajectories getting wider throughout physical time.
        \item A classical optimization approach was implemented as part of the development process; nevertheless, it did not produce different results from the pseudo-inverse approach and instead was not able to keep up with the increase in dimension of the observable space. As a consequence, it was removed.
        \item Spline-radial functions yielded an overall better prediction horizon than the polynomial observables, which opens the question of wether or not better observable structures should be researched.
        \item Finally, higher-degree observables yielded better approximations of the original system, but required a higher than theoretical data library in order to converge when applying the pseudo-inverse.
    \end{itemize}
\end{frame}

\end{document}